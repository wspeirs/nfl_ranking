\documentclass[11pt]{article}

\usepackage{amsmath}
\usepackage{amsthm}
\usepackage{amssymb}
\usepackage{graphicx}
\usepackage{fancyhdr}
\usepackage{nameref}
\usepackage{float}


% these two lines will give you double spacing in the document (only for editing)
%\usepackage{setspace}
%\doublespacing

\usepackage{comment}

\usepackage{url}

\bibliographystyle{plain}

\usepackage[letterpaper,hmargin=1in,vmargin=1.25in]{geometry}

% \topmargin=-0.5in
% \textheight=9.1in
% \oddsidemargin=0in
% \evensidemargin=0in
% \textwidth=6.5in
% \linewidth=6.5in

\title{On the Possibility of Predicting NFL Games Using Rankings}
\author{William R. Speirs}
\date{} % so that it won't show up

% setup for the definitions
\newtheorem{definition}{Definition}

% setup a few commands to make things easier
\newcommand{\bigO}{\mathcal{O}}
\newcommand{\stddev}{\sigma}

% setup a new counter for my definitions
% \newcounter{def}
% \setcounter{def}{1}

% setup a new environment for my definitions
% \newenvironment{definition}
% {\textbf{Definition} \thedef: }{\fbox{ } \addtocounter{def}{1}}


\begin{document}

\maketitle

\small
\textbf{Abstract}. This is a paper.\\
\normalsize


\section{Introduction}

This is the introduction of a paper.

\section{The Predictability of an NLF Game}

Before looking at how accurate a ranking system can be in predicting the outcome of an NFL football game, it is good to start by looking at how predictable NFL games are in general. A lot of NFL games are actually closer than most people would initially assume. This is partially because blow-outs tend to stick in ones mind more than a 3 point win.[CITATION NEEDED?]\footnote{Except of course if it is ``your'' team playing against a big rival.} A quarter of NFL games are decided by 3 points or fewer, and in most seasons only a touchdown (and extra point) decide almost 50\% of NFL games. The percent of games won by a margin of victory or less can be found in Table \ref{table:margin_of_victory}.

\begin{table}[!htb]
\begin{center}
\scriptsize
\begin{tabular}{|c|c|c|c|c|c|c|c|c|c|c|c|c|c|c|c|c|}
\hline
MOV & 2002 & 2003 & 2004 & 2005 & 2006 & 2007 & 2008 & 2009 \\
\hline
1 & 3.9 & 3.5 & 3.9 & 4.7 & 3.1 & 2.3 & 4.3 & 3.1\\
2 & 8.6 & 7.4 & 5.5 & 9.0 & 7.4 & 5.5 & 9.0 & 5.9\\
3 & \textbf{24.6} & \textbf{23.4} & \textbf{23.8} & \textbf{23.4} & \textbf{23.8} & 21.5 & 19.5 & 21.1\\
4 & 28.5 & 28.1 & 27.3 & 30.1 & \textbf{26.9} & \textbf{24.6} & 28.5 & \textbf{25.8}\\
5 & 31.2 & 30.9 & 31.2 & 32.4 & 29.7 & \textbf{26.9} & 31.2 & 28.1\\
6 & 37.9 & 36.7 & 35.5 & 38.3 & 35.9 & 32.4 & 35.2 & 33.2\\
7 & \textbf{49.2} & \textbf{48.4} & 45.3 & 44.5 & 45.7 & 43.0 & 44.9 & 43.0\\
8 & 53.5 & \textbf{51.6} & 47.3 & \textbf{48.0} & \textbf{49.2} & 46.9 & 46.1 & 46.9\\
9 & 54.7 & 52.7 & \textbf{48.0} & \textbf{49.2} & \textbf{50.8} & \textbf{48.4} & \textbf{50.8} & 47.7\\
10 & 60.2 & 57.0 & 57.0 & 56.2 & 57.8 & 53.5 & 57.0 & 53.9\\
14 & 71.1 & 68.0 & 72.3 & 70.7 & 70.3 & 66.4 & 65.6 & 66.0\\
21 & 86.7 & 84.8 & 85.9 & 85.2 & 85.9 & 84.4 & 82.8 & 79.7\\
28 & 95.3 & 93.4 & 96.1 & 94.1 & 94.5 & 92.2 & 93.4 & 91.8\\
35 & 98.4 & 97.3 & 99.6 & 96.9 & 98.8 & 97.7 & 98.0 & 96.1\\
42 & 99.6 & 100.0 & 99.6 & 99.2 & 100.0 & 99.2 & 99.6 & 99.2\\
49 & 100.0 & 100.0 & 100.0 & 100.0 & 100.0 & 100.0 & 100.0 & 99.6\\
56 & 100.0 & 100.0 & 100.0 & 100.0 & 100.0 & 100.0 & 100.0 & 99.6\\
\hline
\hline
MOV & 2010 & 2011 & 2012 & 2013 & 2014 & 2015 & 2016 & 2017 \\
\hline
1 & 3.5 & 3.1 & 5.1 & 5.9 & 4.3 & 3.1 & 7.0 & 1.9\\
2 & 7.0 & 5.9 & 9.0 & 10.6 & 9.4 & 6.2 & 11.7 & 4.7\\
3 & \textbf{25.4} & 19.5 & 22.3 & \textbf{25.4} & 20.7 & \textbf{23.1} & \textbf{24.2} & 21.1\\
4 & 31.6 & \textbf{26.2} & 27.3 & 32.4 & \textbf{24.2} & 27.7 & 28.9 & \textbf{24.6}\\
5 & 35.5 & 30.1 & 30.1 & 34.4 & \textbf{26.9} & 31.6 & 34.4 & 29.7\\
6 & 40.2 & 37.1 & 36.7 & 39.5 & 32.0 & 41.0 & 41.8 & 37.1\\
7 & 47.3 & \textbf{48.8} & 46.9 & \textbf{48.0} & 38.3 & \textbf{51.2} & 52.7 & 43.4\\
8 & \textbf{51.2} & \textbf{51.6} & \textbf{51.2} & \textbf{51.2} & 43.0 & 54.7 & 57.0 & 47.3\\
9 & \textbf{52.0} & 52.3 & \textbf{52.0} & \textbf{52.0} & 44.9 & 55.9 & 59.4 & \textbf{50.4}\\
10 & 57.8 & 57.4 & 58.2 & 58.6 & \textbf{50.4} & 61.3 & 62.5 & 56.2\\
14 & 67.6 & 69.1 & 67.2 & 69.5 & 64.5 & 73.4 & 73.4 & 68.8\\
21 & 83.6 & 81.2 & 83.2 & 84.8 & 82.8 & 85.9 & 90.6 & 84.8\\
28 & 92.6 & 94.1 & 91.8 & 94.5 & 93.0 & 93.8 & 96.5 & 95.7\\
35 & 98.4 & 98.0 & 97.7 & 98.8 & 98.4 & 99.2 & 99.2 & 97.7\\
42 & 99.6 & 98.8 & 99.2 & 99.6 & 99.6 & 100.0 & 100.0 & 99.6\\
49 & 100.0 & 99.6 & 99.6 & 100.0 & 99.6 & 100.0 & 100.0 & 100.0\\
56 & 100.0 & 100.0 & 99.6 & 100.0 & 100.0 & 100.0 & 100.0 & 100.0\\
\hline
\end{tabular}
\caption{Percent of games won by a margin of victory or less.}\label{table:margin_of_victory}
\end{center}
\end{table}

Diving a bit deeper into the numbers, we can use halftime scores as compared to the outcome of the game as a indication of how hard it is to predict the winner of a game. If two teams are very closely matched, it would not be surprising that 50\% of the time the team that is leading at halftime is not necessarily the team that wins. After looking at the data for NFL games, we find that on average 25\% of the games end with the winning team losing at halftime. Table \ref{table:halftime_flip_flop} shows the percent of games played that season where the winner at halftime was the overall loser of the game. Extrapolating, we can say that on average 50\% of the games played in the NFL are between two closely matched teams.

\begin{table}[!htb]
\begin{center}
\scriptsize
\begin{tabular}{|c|c|}
\hline
2000 & 22.58\\
2001 & 22.58\\
2002 & 25.00\\
2003 & 26.17\\
2004 & 24.61\\
2005 & 25.39\\
2006 & 26.56\\
2007 & 23.05\\
2008 & 24.61\\
2009 & 18.75\\
2010 & 25.39\\
2011 & 25.00\\
2012 & 25.39\\
2013 & 25.00\\
2014 & 30.47\\
2015 & 26.17\\
\hline
\textbf{Average} & \textbf{24.80}\\
\hline
\end{tabular}
\caption{Percent of games where the losing team was ahead at halftime.}\label{table:halftime_flip_flop}
\end{center}
\end{table}

Gambling provides us with another measure of how difficult it is to predict the winner of an NFL game. While the point spread established by casinos and bookies does not necessarily represent an adjustment to the scores to make the outcome equal. However, it does represent the betting publics consensus on who will win the game. Spreads are established to promote an equal amount of money being bet on both teams. With all of the money and ``expertise'' that goes into betting on NFL games, one would think that the favorite would almost always win. However, on average the favorite only wins 66\% of the time. Table \ref{table:favorite_wins} shows the complete season-by-season results.

\begin{table}[!htb]
\begin{center}
\scriptsize
\begin{tabular}{|c|c|c|}
\hline
Year & Favorite Won & Favorite Covered\\
\hline
2000 & 64.52 & 45.97\\
2001 & 66.13 & 45.56\\
2002 & 62.50 & 43.36\\
2003 & 67.58 & 48.83\\
2004 & 65.63 & 48.05\\
2005 & 74.22 & 56.25\\
2006 & 59.38 & 42.58\\
2007 & 69.14 & 50.78\\
2008 & 68.36 & 48.05\\
2009 & 69.53 & 47.27\\
2010 & 66.02 & 48.05\\
2011 & 66.41 & 45.70\\
2012 & 64.06 & 46.48\\
2013 & 68.75 & 49.61\\
2014 & 67.58 & 48.44\\
2015 & 63.28 & 46.88\\
\hline
Average & 66.44 & 47.62\\
\hline
\end{tabular}
\caption{Percent of games won by the favorite.}\label{table:favorite_wins}
\end{center}
\end{table}

Taking into account the point spread, we can see that the favorite does not consistently cover the point spread set by the casinos. So while the betting public can identify the favorite more often than not, the magnitude of the victory is harder to identify. This makes sense given the numbers in Table \ref{table:margin_of_victory} and \ref{table:halftime_flip_flop}. On average, a team scores 21.625 points per game,\footnote{$\stddev = 10.395$} so this leaves little room for inaccuracies when predicting the outcome of a game. 

\section{Ranking Systems to Predict Outcomes}

Below are the top raking systems, and their percent correct by year

%
% This is wrong... the spread bests are different
% 

\begin{table}[!htb]
\begin{center}
\scriptsize
\begin{tabular}{|c|l|c|c|}
\hline
Year & \multicolumn{1}{|c|}{System} & \% Correct & Against Spread\\
\hline
2000 & PFZ w/team HFA & 67.5 & 50.0\\
2001 & CPA Retro Rankings & 69.1 & 52.2\\
2002 & Logistic Regression & 68.2 & 57.2\\
2003 & Jeff Self & 68.9 & 50.2\\
2004 & The Sports Report & 79.4 & 55.4\\
2005 & Line (updated) & 72.2 & 00.0\\
2006 & Least Abs. Val Reg & 62.5 & 49.8\\
2007 & Cover81 & 72.0 & 58.2\\
2008 & Jeff Self & 68.4 & 50.6\\
2009 & JFM Power Ratings & 69.2 & 52.1\\
2010 & Cover81 & 66.0 & 51.2\\
2011 & Nationalsportsrankings & 69.7 & 52.3\\
2012 & Bihl Rankings & 71.9 & 50.0\\
2013 & Ironrank.com & 71.4 & 63.4\\
2014 & Dokter Entropy & 71.4 & 50.9\\
2015 & Sagarin Golden Mean & 65.9 & 55.9\\
2016 & Bihl Rankings & 69.4 & 42.9\\
2017 & Bihl Rankings & 71.0 & 53.5\\
\hline
 & \multicolumn{1}{|r|}{\textbf{Average}} & 69.7 & 49.8\\
 & \multicolumn{1}{|r|}{\textbf{Median}} & 69.3 & 51.7\\
\hline
\end{tabular}
\caption{Best Ranking System by Year}\label{table:ranking_year}
\end{center}
\end{table}


\section{Optimal Final Ranking}

Throughout the 17 weeks of an NFL season, many things can happen that alter the performance of a team. For example, in 2007 the New England Patriots went 16-0\cite{ESPN:ne_2007} in the regular season making them one of the best teams in this history of the game. However, in the first game of the 2008 season their MVP quarterback sustained a knee injury that would keep him from playing for the entire 2008 season.\cite{wiki:tom_brady} Their record that season was a still impressive 11-5\footnote{Tied with the Miami Dolphins for first place in the AFC East.}.\cite{ESPN:ne_2008} While the ability of a team is constantly in flux throughout the season because of injuries, trades, and extra curricular activites off the field, there is an expectation that for any given year, comparing two teams once can be considered better than the other. There are plenty of instances where two teams play each other twice in a season with each team taking home a win. However, if the teams were to play each other enough times in a given season, it would be expected that one team would generally win more games than the other.

As such, an optimal ranking for a given year can be constructed such that the fewest number of upsets are allowed. These types of rankings are typically called a Slater Ranking \cite{biometrika_slater} or a Feedback Arc Set \cite{combinatorics_charbit} \cite{discrete_math_alon}, and are defined for the purposes of this paper in Definition \ref{def:slater_ranking}.

\begin{definition}\label{def:slater_ranking}
Given a ranking of the teams in a given NFL season, where $i < j$, team $i$ is ranked as a better team than team $j$. The notation $T_i \rightarrow T_j$ is used to denote team $T_i$ beating team $T_j$. A Slater Ranking is a ranking that minimizes the occurances of $T_j \rightarrow T_i$ in the season.
\end{definition}

Unfortunately, finding a Slater Ranking is an NP-Hard problem \cite{combinatorics_charbit} \cite{discrete_math_alon}. Therefore, constructing a perfectly optimal ranking would require finding the minimum number of upsets after testing $32! \approx 2.63 \times 10^{35}$ different rankings. To make the problem tractible, a greedy algorithm was used after exhausting all possible combinations and permutations of three teams occupying the top three spots in the ranking. The greedy algorithm takes the team with the fewest number of losses, excluding losses to teams already ranked, and assigns that team the next ranking. This resulted in picking the ranking that resulted in the minimum number of upsets after considering 29,760 rankings\footnote{In 2000 and 2001 there were only 31 NFL teams, reducing the number of rankings to 26,970.} for each season. The results of these rankings can be found in Table \ref{table:optimal_ranking}.

\begin{table}[!htb]
\begin{center}
\scriptsize
\begin{tabular}{|c|c|c|c|c|c|c|c|c|c|c|c|c|c|c|c|}
\hline
 & 2000 & 2001 & 2002 & 2003 & 2004 & 2005 & 2006 & 2007 & 2008 & 2009 & 2010 & 2011 & 2012 & 2013 & 2014\\
\hline
$\%$ Correct & $76$ & $76$ & $75$ & $76$ & $77$ & $80$ & $75$ & $79$ & $77$ & $79$ & 73 & 77 & 75 & 74 & 79\\
1 & IND & GB & CLE & STL & PIT & SEA & SD & JAX & TEN & PIT & NYG & GB & SEA & CHI & PIT\\
2 & TAM & OAK & MIA & NWE & NWE & KC & IND & TAM & ATL & HOU & KC & OAK & MIN & PIT & KC\\
3 & STL & BAL & SFO & DAL & NOR & CAR & DEN & ARI & CHI & MIN & NOR & SFO & ATL & IND & SEA\\
4 & MIN & CHI & OAK & PHI & PHI & IND & BAL & NWE & IND & GB & NWE & NOR & DEN & DEN & GB\\
5 & MIA & STL & TEN & KC & IND & DEN & NWE & DAL & PIT & SD & PIT & DET & SFO & SEA & NWE\\
6 & BUF & SFO & PIT & GB & BAL & JAX & CHI & GB & BAL & IND & ATL & CHI & GB & KC & DEN\\
7 & DET & PHI & IND & DEN & NYJ & NWE & BUF & NYG & MIN & BAL & BAL & ATL & CHI & SFO & ARI\\
8 & NYG & PIT & PHI & IND & BUF & PIT & NYJ & WAS & HOU & DAL & TAM & PIT & NWE & ARI & SD\\
9 & TEN & TAM & TAM & TEN & SD & CIN & TEN & SD & CAR & NOR & CHI & NWE & HOU & TEN & IND\\
10 & PHI & NYJ & GB & MIA & CIN & CHI & CIN & IND & NYG & PHI & NYJ & SD & IND & CAR & CIN\\
11 & BAL & MIA & ATL & BUF & CLE & TAM & PIT & TEN & NWE & DEN & GB & DEN & TEN & STL & BAL\\
12 & PIT & NWE & BAL & CAR & GB & ATL & KC & HOU & MIA & NYG & MIN & MIN & DET & NOR & DET\\
13 & NYJ & IND & DEN & NYJ & DAL & MIN & MIA & CAR & ARI & NWE & BUF & CAR & STL & NWE & MIA\\
14 & GB & ATL & NYG & BAL & WAS & DET & JAX & DET & DAL & ATL & CLE & BAL & BUF & CIN & BUF\\
15 & CHI & NOR & NWE & JAX & SEA & BAL & NOR & DEN & CIN & MIA & MIA & CIN & ARI & GB & NYJ\\
16 & NWE & BUF & JAX & NOR & MIN & CLE & PHI & PIT & WAS & TEN & CIN & HOU & MIA & BAL & MIN\\
17 & DAL & CAR & KC & HOU & HOU & GB & GB & BUF & PHI & SFO & CAR & TEN & NYJ & NYJ & CHI\\
18 & WAS & MIN & NYJ & TAM & JAX & MIA & NYG & CLE & CLE & ARI & DET & CLE & JAX & TAM & SFO\\
19 & CIN & NYG & BUF & NYG & DET & SD & DAL & SEA & JAX & CHI & PHI & JAX & BAL & ATL & DAL\\
20 & JAX & KC & WAS & WAS & CHI & NYG & HOU & PHI & GB & SEA & SFO & TAM & CIN & BUF & PHI\\
21 & ARI & SEA & SEA & ATL & NYG & WAS & CLE & MIN & SFO & JAX & SEA & IND & WAS & MIA & NYG\\
22 & DEN & DEN & STL & SEA & ATL & NOR & MIN & CHI & BUF & STL & STL & SEA & DAL & CLE & STL\\
23 & OAK & SD & SD & CHI & DEN & NYJ & CAR & OAK & SEA & NYJ & ARI & ARI & NYG & JAX & HOU\\
24 & CLE & DAL & ARI & CLE & CAR & BUF & ATL & KC & STL & BUF & SD & PHI & CLE & HOU & WAS\\
25 & NOR & WAS & HOU & PIT & KC & OAK & TAM & CIN & SD & CAR & DAL & NYG & PIT & SD & OAK\\
26 & ATL & ARI & DAL & ARI & OAK & DAL & WAS & BAL & DEN & CIN & WAS & DAL & PHI & DAL & JAX\\
27 & CAR & JAX & CAR & CIN & TEN & PHI & SEA & NYJ & NYJ & DET & IND & STL & TAM & NYG & CLE\\
28 & SFO & CIN & CIN & SD & MIA & ARI & STL & MIA & NOR & CLE & JAX & MIA & CAR & MIN & TEN\\
29 & KC & TEN & MIN & OAK & STL & TEN & ARI & STL & OAK & KC & DEN & NYJ & SD & PHI & ATL\\
30 & SEA & CLE & DET & MIN & TAM & SFO & SFO & NOR & TAM & WAS & HOU & BUF & OAK & DET & CAR\\
31 & SD & DET & NOR & SFO & SFO & STL & OAK & ATL & KC & OAK & TEN & KC & KC & WAS & NOR\\
32 & & & CHI & DET & ARI & HOU & DET & SFO & DET & TAM & OAK & WAS & NOR & OAK & TAM\\
\hline
\end{tabular}
\caption{Optimal ranking of teams by season}\label{table:optimal_ranking}
\end{center}
\end{table}


\begin{itemize}
\item Talk about NP-Hard problem, and our pseudo-solution
\item Compute optimal ranking for each year
\item Compare with \\% of games that ``could have been predicted''
\end{itemize}

\section{Optimal Week-by-Week Ranking}

\begin{itemize}
\item Is this an NP-Hard problem?
\item Compute perfect ranking each week (for each year) with the fewest moves from week-to-week
\item Compare to a simple score-based ranking
\end{itemize}


\newpage

\bibliography{paper}

%\newpage
%\appendix
%\thispagestyle{plain}
%\onecolumn

% This will put "Appendix" before the letter... it's a nice touch, just watch when you reference things
%\newcommand{\appsection}[1]{\let\oldthesection\thesection
%  \renewcommand{\thesection}{Appendix \oldthesection}
%  \section{#1}\let\thesection\oldthesection}
  
%\appsection{}

\end{document}
